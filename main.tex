% !TEX root = main.tex
\input{../.tex-common/preamble.tex}

\begin{document}

\begin{center}
    Программа курса лекций \\
    \textbf{<<Физические основы фотоники и нанофотоники>>}
\end{center}

\begin{enumerate}
\item \textbf{Основные определения фотоники} как технологии генерации и преобразования излучения с использованием фотона в качестве квантовой единицы. Области науки, входящие в ФОТОНИКУ.

\item \textbf{Квантовая теория теплового излучения}. Формула Планка. Излучение чёрного тела. Индуцированное и спонтанное излучение. Связь коэффициентов Эйнштейна для спонтанного и вынужденного излучения.

\item \textbf{Уравнения генерации лазера}. Классическое описание (вероятностный подход, уравнения Статца-Де Марса), полуклассическое (самосогласованные уравнения, уравнение Ван дер Поля), понятия о квантовых уравнениях. Скоростные уравнения: пороговые условия генерации, стационарный и модуляции добротности режимы. Уравнения Ван дер Поля лазерной генерации.

\item \textbf{Оптические резонаторы}. Открытый резонатор. Основные параметры резонатора: добротность, число Френеля, критерий устойчивости. Параметры лазерных пучков: расходимость, фактор качества $M^2$.

\item \textbf{Гауссов пучок как решение волнового уравнения} в параксиальном приближении. Моды высшего порядка. Понятие лучевых матриц. Методы расчёта резонаторов: на основе дифракционного интеграла и $ABCD$-закона преобразования комплексного параметра. Обобщённый двухзеркальный резонатор, области устойчивости.

\item \textbf{Спонтанное и стимулированное излучение}. Атомные переходы в конденсированной среде. Форма линии. Коэффициенты поглощения и усиления. Инверсная населённость.

\item \textbf{Когерентность электромагнитного излучения}. Многолучевая интерференция. Лазерные интерференционные зеркала.

\item \textbf{Оптические волоконные световоды}. Волоконно-оптические линии связи (ВОЛС). Материальная и волноводная дисперсия в световодах. Одно- и многомодовые волокна. Микроструктурные волокна (фотонные кристаллы). Виды потерь в волокне. Типы волокон. Активные волокна. Датчики на основе оптоволокна.

\item \textbf{Полупроводниковые источники лазерного излучения}. Твёрдые растворы соединений. Гомо- и гетеропереходы. Квазиуровни Ферми. Энергетическая зонная диаграмма лазерного диода.

\item \textbf{Электронные волны де Бройля и зонная диаграмма}. Прямые и непрямые переходы. Неравновесные состояния. Условия генерации. Кванторазмерные эффекты в п/п лазерах. Каскадные лазеры. РОС лазерные диоды. Светодиоды как новое поколение источников света.

\item \textbf{Фотоприёмники --- физические основы работы}. ,Фотодиоды, вольт-амперная характеристика, лавинный фотодиод. Шумы полупроводниковых приёмников излучения. ФЭУ. ЭОП. Матричные фотоприёмники.

\item \textbf{Нелинейная оптика. Генерация второй гармоники}. Параметрическая генерация. Понятие фазового синхронизма. Скалярный и векторный синхронизм. Укороченные уравнения. Генерация гармоник высокого порядка.

\item \textbf{Синхронизация мод в лазерах}, методы синхронизации. Сверхсильные световые поля. Нелинейно-оптические эффекты в лазерном поле. Пико- и фемтосекундные импульсы излучения. Спектрально-ограниченные импульсы. <<Чирп>> частоты. Волоконно-оптические компрессоры. Понятие о синхронизме в нелинейной оптике.

\item \textbf{Распространение сверхкоротких лазерных импульсов в оптических средах:} линейной дисперсионной среде; усиливающей среде; нелинейной среде с керровской нелинейностью; через частотный фильтр. Описание лазерных импульсов. Распространение волнового пакета в дисперсионной линейной среде с дисперсией вида \(n=n_0+\alpha\omega\). Керровская нелинейность. Поведение волнового пакета в нелинейной среде \(n(I)=n_0+\bar{n}_2 I\). Прохождение волнового пакета через частотный фильтр.

\item Характерные интенсивности лазерного поля. \textbf{Методы формирования сверхкоротких импульсов}, измерения параметров. Петаваттные лазерные комплексы.

\item \textbf{Волоконные лазеры}. Активные волоконные среды. Концентрационное тушение. Брэгговские волоконные зеркала. Нелинейные явления в волоконных лазерах.

\item \textbf{Адаптивная оптика}. Датчики Шака-Гартмана. Нелинейные адаптивные оптические системы на эффекте рассеяния Мандельштама-Бриллюэна. 3-x и 4-x частотное взаимодействие.

\item \textbf{Оптическая гирометрия. Эффект Саньяка}. Лазерные оптические гироскопы. Волоконно-оптические гироскопы. Эффекты невзаимности встречных волн. Атомно-лучевая гирометрия. Волны де Бройля. Лазерное охлаждение. Доплеровский метод. Магнитооптические ловушки.
\item \textbf{Широкозонные и узкозонные сенсоры} для фотоприёмников и приборов ночного видения.
\end{enumerate}


\end{document}